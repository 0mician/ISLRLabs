% Created 2016-05-09 Mon 09:14
\documentclass[11pt, a4paper]{article}
\usepackage[utf8]{inputenc}
\usepackage[T1]{fontenc}
\usepackage{fixltx2e}
\usepackage{graphicx}
\usepackage{longtable}
\usepackage{float}
\usepackage{wrapfig}
\usepackage{soul}
\usepackage{textcomp}
\usepackage{marvosym}
\usepackage{wasysym}
\usepackage{latexsym}
\usepackage{amssymb}
\usepackage{hyperref}
\tolerance=1000
\usepackage{minted}
\usepackage[utf8]{inputenc}
\usepackage[english]{babel}
\usepackage{graphicx}
\usepackage[left=2.35cm, right=3.35cm, top=3.35cm, bottom=3.0cm]{geometry}
\usepackage{titling}
\providecommand{\alert}[1]{\textbf{#1}}

\title{Statistical methods for bioinformatics \linebreak Beyond linearity: case study (de Vijver et al.)}
\author{Cedric Lood}
\date{\today}
\hypersetup{
  pdfkeywords={},
  pdfsubject={},
  pdfcreator={Emacs Org-mode version 7.9.3f}}

\begin{document}

\maketitle


\graphicspath{ {figures/} }
\setlength{\droptitle}{-5em} 
\setlength{\parindent}{0cm}

\section{Exercise 1}
\label{sec-1}
\subsection{Part a}
\label{sec-1-1}

This is straightforward by taking the coefficients $a_1=\beta_0,
b_1=\beta_1, c_1=\beta_2, d_1=\beta_3$
\subsection{Part b}
\label{sec-1-2}

Since we are looking at $x>\xi$, we have the form:

\begin{itemize}
\item $f(x)=\beta_0 + \beta_1 x + \beta_2 x^2 + \beta_3 x^3 + \beta_4 (x - \xi)^3$
\end{itemize}

We can distribute the cube and get:

\begin{itemize}
\item $f(x)=\beta_0 + \beta_1 x + \beta_2 x^2 + \beta_3 x^3 + \beta_4 (x^3 - 3 x^2 \xi + 3 x \xi^2 - \xi^3)$
\end{itemize}

The expression can be re-arranged to highlight the predictors
coefficients' values:
 
\begin{itemize}
\item $f(x)=(\beta_0 - \beta_4 \xi^3) + (\beta_1 + 3 \beta_4 \xi^2) x +(\beta_2 - 3 \beta_4 \xi) x^2 + (\beta_3 + \beta_4) x^3$
\end{itemize}

Combining with the answer of Part a, we now see that $f(x)$ is a
piecewise polynomial
\subsection{Part c}
\label{sec-1-3}


Showing that the 2 pieces are connected continuously can be done by
solving both in $\xi$:

\begin{itemize}
\item $f_1(\xi)=\beta_0 + \beta_1 \xi + \beta_2 \xi^2 + \beta_3 \xi^3$
\item $f_2(\xi)=(\beta_0 - \beta_4 \xi^3) + (\beta_1 + 3 \beta_4 \xi^2) \xi  +(\beta_2 - 3 \beta_4 \xi) \xi^2 + (\beta_3 + \beta_4) \xi^3$
\end{itemize}

When distributing the terms in $f_2(\xi)$, one obtains that the terms
in $\beta_4$ cancel each other, and that $f_2(\xi)=\beta_0 + \beta_1 \xi + \beta_2 \xi^2 + \beta_3 \xi^3=f_1(\xi)$
\subsection{Part d}
\label{sec-1-4}

For this, we need the first order derivatives of $f_1(x)$ and $f_2(x)$
\begin{itemize}
\item $f_1'(x)=\beta_1 + 2\beta_2 x + 3\beta_3 x^2$
\item $f_2'(x)=\beta_1 + 3 \beta_4 x^2 + 2 (\beta_2 - 3 \beta_4 x) x + 3 (\beta_3 + \beta_4) x^2$
\end{itemize}

Solving the equations above for $x=\xi$:

\begin{itemize}
\item $f_1'(\xi)=\beta_1 + 2\beta_2 \xi + 3\beta_3 \xi^2$
\item $f_2'(\xi) = \beta_1 + 3\beta_4 \xi^2 + 2(\beta_2 - 3\beta_4 \xi) \xi + 3(\beta_3 + \beta_4) \xi^2$
\item $f_2'(\xi) = \beta_1 + 3\beta_4 \xi^2 + 2\beta_2 \xi - 6\beta_4 \xi^2 + 3\beta_3 \xi^2 + 3\beta_4 \xi^2$
\item $f_2'(\xi) = \beta_1 + 2\beta_2 \xi + 3\beta_3 \xi^2 + 3\beta_4 \xi^2 + 3\beta_4 \xi^2 - 6\beta_4 \xi^2$
\item $f_2'(\xi) = \beta_1 + 2\beta_2 \xi + 3\beta_3 \xi^2$
\end{itemize}

Hence, $f_1'(\xi)=\beta_1 + 2\beta_2 \xi + 3\beta_3 \xi^2=f_2'(\xi)$
\subsection{Part e}
\label{sec-1-5}

We can take the second order derivatives of $f_1(x)$ and $f_2(x)$,
and solve in $\xi$ to verify if the transition is continuous:

\begin{itemize}
\item $f_1''(x) = 2\beta_2 + 6\beta_3 x$
\item $f_2''(x) = 2(\beta_2 - 3\beta_4 x) + 6(\beta_3 + \beta_4) x = 2\beta_2 + 6\beta_3 x$
\end{itemize}

Hence, $f_1''(\xi) = 2\beta_2 + 6\beta_3 \xi=f_2''(x)$ \\

Combining this with the previous parts, we have shown that $f(x)$ is
indeed a cubic spline, where the piecewise functions $f_1(x)$ and
$f_2(x)$ are indeed connected, and where the first order and second
order derivatives are smoothly connected.
\section{Exercise 9}
\label{sec-2}
\subsection{part a}
\label{sec-2-1}
\subsection{part b}
\label{sec-2-2}
\subsection{part c}
\label{sec-2-3}
\subsection{part d}
\label{sec-2-4}
\subsection{part e}
\label{sec-2-5}
\subsection{part f}
\label{sec-2-6}

\end{document}
